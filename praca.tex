Zastosowanie głębokich sieci neuronowych do obrazu


Przedmowa

W niniejszej pracy (dalej nazywana po prostu pracą) zakłada się, że czytelnik zaznajomiony jest z podstawowymi pojęciami dotyczącymi sieci neuronowych (dalej często nazywanych po prostu sieciami) - neuron, warstwa ukryta, funkcja aktywacji, propagacja wsteczna itp. W pracy zostanie przedstawione działanie zaawansowanych sieci głębokich na przykładzie sieci 'deep belief network'. W związku z brakiem polskiego nazewnictwa w pracy, poza wstępem, będzie używana nazwa 'sieć głęboka'. We wstępie, aby zapobiec nieścisłościom typ sieci będzie podawany jawnie.
Integralną częścią pracy jest aplikacja DipNet będąca środowiskiem do tworzenia, uczenia oraz testowania sieci. Aplikacja ta dostępna jest w repozytorium git pod adresem https://github.com/KolodzinskiKrzysztof/dipNet

Wstęp

Głębokie sieci neuronowe

Jak podaje Wikipedia, głęboka sieć neuronowa to taka, w której skład wchodzi wiele warstw ukrytych. Do definicji tej pasują m. in. :popularne w dziedzinie obrazów konwolucujne głębokie sieci neuronowe (convolutional deep neural networks - 1989r.), sieci długiej pamięci krótkotrwałej [1 - tłumaczenie własne] (Long short term memory - 1997r.) oraz wspomniane wcześniej i będące głównym tematem pracy sieci typu 'deep belief network'.

Problemy z sieciami głębokimi

Istnieje wiele powodów, przez które głębokie sieci neuronowe nie były wykorzystywane. Do niedawna głównym powodem był kosz obliczeń - głębokie sieci neuronowe potrzebują dużej mocy obliczeniowej, aby działać sprawnie. Dziś wydajność sprzętowa jednostek obliczeniowych jest nieporównywalnie większa i o wiele łatwiej stosować takie sieci.
Drugim, najczęstszym powodem jest trudność uczenia i skłonność do przeuczenia (overfitting) - propagacja wsteczna i metoda gradientu nie są odpowiednimi narzędziami do uczenia wielowarstwowych sieci. Przy sieciach głębokich metody te mają tendencje do zatrzymywania się w minimach lokalnych, a propagacja błędu z kolejną warstwą w tył jest coraz mniejsza co pozwala sprawnie uczyć tylko końcowe warstwy, zawodząc przy początkowych.

Deep belief networks

Sieci będące głównym tematem tej pracy są złożeniem kilku typów różnych sieci. Głównym jej elementem są warstwy ograniczonej maszyny Boltzmana.

Ograniczona maszyna Boltzmana (restricted Boltzman machine):

Ograniczona maszyna Boltzmana jest siecią neuronową złożoną z dwóch grup neuronów - widocznych i ukrytych. 
